\documentclass{article}

\title{CARMEN map file format definition}
\author{Mike Montemerlo and Nicholas Roy and Sebastian Thrun}
\date{Revised: October 3, 2001}

\begin{document}

\maketitle

\begin{verbatim}
Version: 1.0

Map files in CARMEN store evidence grid representations of 2-D
worlds as well as other information related to robot navigation and
localization. The format of these data files is a tagged file format,
similar in spirit to the TIFF standard for image files. This is the
official description of the CARMEN map file format.

Filenames:

All CARMEN map files must end with the extension ".map". Map files
may be compressed using gzip, but they must end with the extension
".map.gz".

Overall file format:

The map file may start with a arbitrary number of lines beginning with
a # (hash). These lines are comments and are disregarded by programs
that read the maps. They should be used to make basic information
about the map human-readable.

The first line of the file that does not start with a # (hash)
signifies the beginning of the binary map data. This data consists of
an arbitrary number of tagged "chunks".

A given CARMEN program may not know how to interpret every type of
chunk in a map file. Programs should simply skip over any chunks that
have types they cannot recognize. This will minimize problems with
compatibility of different programs and maps.

CHUNK DESCRIPTION:

Byte 1 : Chunk type - unsigned byte

This byte describes the type of the next chunk.  Here are the
currently recognized chunk types:

0 - file creation information
1 - evidence grid map
2 - off limits area description
3 - named positions
4 - expected distances
5 - laser scans

Byte 2-5 : Chunk size - unsigned int

This is the number of bytes of the chunk, not including the chunk type
and the chunk size variables.

Byte 6-15: ASCII chunk description - 10 ASCII chars

This is an ASCII description of the chunk type.  It was added to make
the files easier to debug in an editor.

The following ASCII descriptions correspond with the standard chunk
types.

0 - "CREATOR   "
1 - "GRIDMAP   "
2 - "OFFLIMITS "
3 - "PLACES    "
4 - "EXPECTED  "
5 - "LASERSCANS"

Bytes 16 - ??: (length specified by chunk size above)

This is the binary data that makes up the individual chunk.

The format of the binary data for each chunk type will be described in
turn.

CREATOR CHUNK DESCRIPTION:

Bytes 1 - 10 : userid - ASCII chars

This is the userid of the account which created the map file.

Bytes 11 - 18 : creation date - long int

This is time / date of creation stored as a number of seconds. This is
easy to change into a real date using UNIX functions.

Bytes 19 - 98 : map description - 80 ASCII chars

General description of the map.

GRIDMAP CHUNK DESCRIPTION:

Bytes 1 - 4 : map columns - 1 unsigned byte

Number of columns in the map

Bytes 5 - 8 : map rows - 1 unsigned bytes

Number of rows in the map

Bytes 9 - 12 : map resolution - 1 float

Number of meters per grid cell.  The resolution must be the same
in x and y.

Bytes 13 - ?? : map data - rows * columns floats

The values of all evidence grid cells.  The cells vary between 0 for
empty to 1 for occupied, and -1 for unknown.  The data is in row
order. (i.e. the first row values followed by the second row values, etc)

OFFLIMITS CHUNK DESCRIPTION:

The off-limits areas can be either off-limits points, or off-limits
rectangles.  The points are specified first, then the rectangles.

Bytes 1 - 4 : number of off-limits points N - unsigned int
Bytes 5 - ?? : off-limits points - N * 4 * 2 unsigned ints
x1 y1 x2 y2 .... Each x and y are unsigned int representing the
off-limits cell coordinates.

4 more bytes : number of off-limits rectangles M - unsigned int
Bytes ?? - ?? : off-limits rectangles - M * 4 * 4 unsigned ints
r1x1 r1y1 r1x2 r1y2 r2x1 r2y1 r2x2 r2y2 ... Each x and y are unsigned
ints representing the coordinates of the off-limits rectangle.

PLACES CHUNK DESCRIPTION:

Bytes 1 - 4: number of named points in the map - unsigned int

Bytes 5 - ?? : description of each location

location bytes 1 - 4 : position type - unsigned int

Type 0 - named position (x and y)
Type 1 - named pose (x, y, and theta)
Type 2 - localization initialization position (x, y, theta,
uncertainties)

location bytes 5 - 8 : number of bytes for the position - unsigned int

Type 0 - 8 bytes
Type 1 - 12 bytes
Type 2 - 24 bytes

location bytes 9 - ?? : position data

Type 0 - 8 bytes - float x, float y
Type 1 - 12 bytes - float x, float y, float theta
Type 2 - 24 bytes - float x, float y, float theta, float xstd, float
ystd, float zstd

EXPECTED CHUNK DESCRIPTION:

This will be filled in later.

LASERSCANS CHUNK DESCRIPTION:

Bytes 1 - 4 : Number of laser scans in the chunk - unsigned int

Laser scan

Bytes 1 - 4 : x position of laser - float
Bytes 5 - 8 : y position of laser - float
Bytes 9 - 12 : theta position of laser - float
Bytes 13 - 16 : number of laser readings
Bytes 17 - ?? : laser readings - N floats    
Laser readings are all in m.
\end{verbatim}
\end{document}
